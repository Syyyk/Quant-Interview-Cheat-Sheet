\documentclass[UTF8]{ctexart}
\usepackage{graphicx}
\usepackage{amsmath}
\usepackage{amssymb}
\usepackage{graphicx}
\usepackage{geometry}
\geometry{left=3.0cm,right=3.0cm,top=3.5cm,bottom=2.5cm}
\everymath{\displaystyle}
\usepackage{float}
\usepackage{setspace}
\setstretch{1.25} 
\usepackage{float}
\usepackage{bookmark}
\newtheorem{thm}{定理}[section]

\begin{document}

\section{Probability and Stochastic Process}

\subsection{Probability}

$\operatorname{Var}X = \mathbb{E}[(X-\mathbb{E}X)^2] = \mathbb{E}[X^2]-(\mathbb{E}X)^2$.
$\operatorname{Var}(aX+b) = a^2 \operatorname{Var}X$.

$\operatorname{Cov}(X,Y) = \mathbb{E}[(X-\mathbb{E}X)(Y-\mathbb{E}Y)]
	= \mathbb{E}[XY]-\mathbb{E}[X]\mathbb{E}[Y]$.
$\operatorname{Cov}(aX+b,Y) = a\operatorname{Cov}(X,Y)$.

$\operatorname{Var}(X+Y)=\operatorname{Var}(X)+\operatorname{Var}(Y)+2\operatorname{Cov}(X,Y)$.

Cov Matrix for $X=(X_1,\dots,X_n)^T$: $\operatorname{Cov}(X) = (\operatorname{Cov}(X_i,X_j))_{i,j}$.

$\operatorname{Corr}(X,Y)=\frac{\operatorname{Cov}(X,Y)}{\sqrt{\operatorname{Var}X \operatorname{Var}Y}}$

矩母函数 Moment generating function: For rv $X$, MGF is $M_X(t)=\mathbb{E}[e^{tX}]$.
For vec-valued rv $\mathbf{X}$, $M_X(\mathbf{t})=\mathbb{E}[e^{\langle \mathbf{t},\mathbf{X} \rangle }]$.
可以生成$n$阶矩 $\mathbb{E}[X^n]=\frac{d^n}{dt^n}M_X \Big| _{t=0}$. 矩母函数相同说明分布相同.

特征函数 Ch.f.: For rv $X$, $\varphi_X(t)=\mathbb{E} [\exp(itX)]$.
For vec-valued rv $\mathbf{X}$, $\varphi_X(\mathbf{t})=\mathbb{E}[e^{i \langle \mathbf{t},\mathbf{X} \rangle }]$.

\noindent \textbf{一些分布} \par 
Binomial$(p, n)$: $P_X(k)=C_n^k p^k (1-p)^{n-k}$. 期望$np$, Var $np(1-p)$. \par 
Poisson$(\lambda)$: $P_X(k)=e^{-\lambda}\frac{\lambda^k}{k!}$. 期望$\lambda$, Var $\lambda$.\par 
Normal$(\mu, \sigma^2)$: $f_X(x)=\frac{1}{\sqrt{2\pi\sigma^2}}\exp(-\frac{(x-\mu)^2}{2\sigma^2})$. 
偏度为0, 峰度为3. MGF: $\exp (\frac12 u^2 t)$.\par 
Exponential$(\lambda)$: $f_X(x) = \lambda e^{-\lambda x}(x\geq 0)$. 期望$\frac{1}{\lambda}$, Var $\frac{1}{\lambda^2}$.


\noindent \textbf{Multivariate Gaussian} $\mathcal{N}(\mu,\Sigma)$

定义: Random vec $X$是$d$维Gaussian Vec 若对任意$t\in\mathbb{R}^d$, $\langle t,X \rangle $都是Gaussian.

$n$个正态分布拼起来不一定是Gaussian vec, 但$n$个独立正态分布拼起来一定是(独立正态分布的和一定是正态分布).
反例: $X\sim\mathcal{N}(0,1), Y=\varepsilon X$ where $\varepsilon=\pm 1$各一半概率且与$X$独立. $X+Y$不是正态. 

给定vec $\mu$和半正定对称阵$\Sigma$, 存在Gaussian Vector $X$ with mean $\mu$ and Cov $\Sigma$.
如果$\Sigma$是满秩的, 则存在density
$f(\mathbf{x}) = (2 \pi)^{-d / 2} \operatorname{det}(\boldsymbol{\Sigma})^{-1 / 2} \exp \left(-\frac{1}{2}(\mathbf{x}-\boldsymbol{\mu})^{\mathrm{T}} \boldsymbol{\Sigma}^{-1}(\mathbf{x}-\boldsymbol{\mu})\right)$,

MGF  $\exp \left( \mu^T t + \frac{1}{2} t^T \Sigma t \right)$.
ChF $\exp \left( i \mu^T t - \frac{1}{2} t^T \Sigma t \right)$.

\noindent \textbf{条件概率条件期望}

条件概率: $\mathbb{P}(A|B)=\frac{\mathbb{P}(A\cap B)}{\mathbb{P}(B)}$.
贝叶斯公式: $\mathbb{P}(A|B) = \mathbb{P}(B|A) \frac{\mathbb{P}(A)}{\mathbb{P}(B)}$.

条件期望:
$\mathbb{E}[X|\mathcal{F}]$是一个随机变量使得
(1)是$\mathcal{F}$可测的;
(2)$\mathbb{E}[\mathbb{E}[X|\mathcal{F}]1_A]=\mathbb{E}[X 1_A], \forall A \in\mathcal{F}$.

对两个rv $X,Y$, 有joint density $f_{X,Y}(x,y)$, 则Marginal density $f_X(x)=\int_{\mathbb{R}} f_{X,Y}(x,v)dv$.
条件密度 $f_{X|Y}(x|y) = \frac{f_{X,Y}(x,y)}{f_Y(y)}$,
条件期望 $\mathbb{E}[X|y]=\int_{\mathbb{R}}xf_{X|Y}(x|y)dx$.
$\mathbb{E}[g(X)|Y]=h(Y)$ where
$h(y) = \frac{\int g(x)f_{X,Y}(x,y)dx}{\int f_{X,Y}(x,y)dx}$.

独立性引理. 若$X, Y$独立, 则$\mathbb{E}[\varphi(X,Y)|Y]=h(Y)$ where
$h(y)=\mathbb{E}[\varphi(X,y)]$.
更一般地, $X_1,\dots,X_k$是$\mathcal{G}$可测的, $Y_1,\dots,Y_l$独立于$\mathcal{G}$,
则$\mathbb{E}[f(X_1,\dots,X_k,Y_1,\dots,Y_l)|\mathcal{G}]=g(x_1,\dots,x_k)$
where $g(x_1,\dots,x_k)=\mathbb{E}[f(x_1,\dots,x_k,Y_1,\dots,Y_l)]$. \\

\noindent \textbf{一些命题}
\begin{itemize}

	\item $F$递增+右连续+$[0,1]$,即可定义一个分布函数.
	      如果存在非负$f$使$F(t)=\int_{-\infty}^t f(x)dx$, 则分布有density.

	\item $X$有pdf $f_X$, $Y=g(X)$其中$g$可微且严格单调, 则$f_Y(y)=f_X(g^{-1}(y))\frac{d}{dy}g^{-1}(y)$.

	\item $X\sim \mathcal{N}(\mu,\sigma^2)$,
	      则$\mathbb{E}[|X|]=\mu\left[2 \Phi\left(\frac{\mu}{\sigma}\right)-1\right]+
		      \frac{2 \sigma}{\sqrt{2 \pi}} \exp \left\{-\frac{\mu^{2}}{2 \sigma^{2}}\right\}$.
	      其中$\Phi$为std norm CDF.

	\item 在$[0,1]$的均匀分布下取$n$个值, 最大值的期望是$\frac{n}{n+1}$, 最大值的密度
	      $f_X(x) = n(1-x)^{n-1}$.

	\item Distribution of $X+Y$ when $X\sim f,Y\sim g$ are independent:
	      $\mathbb{P} (X+Y\leq z)=\int\mathbb{P}(x\leq z-y)g(y)f(y)dy$,
	      density of $X+Y$ is $h(z)=\int f(z-y)g(y)dy$.

	\item 顺序统计量 $k$th order statistic: the $k$th smallest value in a sample of $n$ from a random distribution.
	      There is $\mathbb{P}(X_{(k)}\leq x)=\sum_{j=k}^n F(x)^j (1-F(x))^{n-j}$.

\end{itemize}

大数定律:
$X_1,\dots,X_n$ 是 iid rv 序列, 期望$\mu$有限.
令 $S_n = X_1+\dots+X_n$, 则
(弱) $\frac{S_n}{n}\to\mu$ in probability i.e.
$\lim_{n\to\infty}\mathbb{P}(|\frac{S_n}{n}-\mu|>\varepsilon)=0$.
(强) Almost surely $\lim_{n\to\infty}\frac{S_n}{n}=\mu$.

中心极限定理 Central Limit Theorem:
$X_1,\dots,X_n$ 是 iid rv 序列, 期望$\mu$和方差$\sigma^2$有限,
令 $S_n = X_1+\dots+X_n$, 则 $Y_n=\frac{S_n-\mu n}{\sigma\sqrt{n}}$
converges in distribution to std norm.


\subsection{Stochastic Process}

\noindent \textbf{鞅和停时}

$(X_n)$ or $(X_t)$是鞅, $T$是停时, 则$(X_{n\wedge T})$ or $(X_{t\wedge T})$是鞅.

[Optional Stopping Thm] $(X_n)$ u.i. sub-mtg, $S\leq T$是停时, 则$\mathbb{E}[X_T|\mathcal{F}_S]\geq X_S$.
$(X_t)$ u.i. mtg, $S\leq T$是停时, 则$\mathbb{E}[X(T)|\mathcal{F}(S)]=X(S)$.

Rk. 一致可积u.i.即$\forall \varepsilon, \exists K$ s.t. $\mathbb{E}[|X_n|1_{|X_n|\geq K}]<\varepsilon (\forall n)$.
对离散sub-mtg, 一致可积$\Leftrightarrow$ convergence a.s. and in $L^1$  $\Leftrightarrow$ convergence in $L^1$.
对连续mtg, 一致可积$\Leftrightarrow$ convergence in $L^1$  $\Leftrightarrow$ $\exists Y\in L^1, X(t)=\mathbb{E}[Y|\mathcal{F}(t)]$.


\noindent \textbf{Brownian Motion 性质} \par
Scaling invariance: $\frac{1}{a} B(a^2t)$是BM. 可得first passage time $T_a=a^2 T_1$ in distr. \par
Time inversion: $X(t)=tB(\frac{1}{t})(t>0),X(0)=0$是BM. 可得 $\lim_{t\to\infty}\frac{B(t)}{t}=0$ a.s. \par
Reflection Principle: $T$是停时, $B_t1_{t\leq T}+(2B_T-B_t)1_{t>T}$是BM. \par
$B(t), B(t)^2-t$是鞅. 指数鞅 $X(t)=\exp(\lambda B(t)-\frac12 \lambda^2 t)$对$\lambda\in\mathbb{R}$. \par
Arcsine law: $L=\sup \{t\in[0,1]:B(t)=0\}$ is distributed as $\mathbb{P}(L\leq x)=\frac{2}{\pi}\arcsin\sqrt{x}$. \par
Wald's Lemma: 停时$T$有$\mathbb{E}T <\infty$, 则$\mathbb{E}[B(T)]=0$, $\mathbb{E}[B(T)^2]=\mathbb{E}T$. \par
Law of the Iterated Logarithm: $\limsup _{t \rightarrow \infty} \frac{B(t)}{\sqrt{2 t \log \log (t)}}=1$ a.s. \par



\noindent \textbf{Stochastic Calculus}

二次变差 $[f,f](T)=\lim_{|\Pi|\to 0} \sum_{j=0}^{n-1}(f(t_{j+1})-f(t_j))^2$. 有 $[W,W](t)=t$ a.s.

Ito Isometry: $\mathbb{E}\left[ (\int_0^T X_t dW_t)(\int_0^T Y_t dW_t) \right]
	=\mathbb{E}\left[ \int_0^T X_t Y_t dt \right]$.

$\mathbb{E}[W_t^2]=t, \mathbb{E}[W_t^4]=3t^2$.
$\int_0^t W_s ds \sim \mathcal{N}(0,\frac13 t^2)$,
$\int_0^t f(s) dW_s \sim \mathcal{N}\left(0,\int_0^t f(s)^2 ds\right)$.

$\int_0^T W_tdW_t = \frac12 (W_T^2-T)$ (apply Ito's lemma to $W_t^2$).

Ito过程: 由$dX(t)=\Delta(X(t),t)dW(t)+\Theta(X(t),t)dt$形式SDE决定的随机过程$X(t)$为Ito过程.
特别地, 若$\Delta$只关于$t$, 则有$[X,X](t)=\int_0^t \Delta(u)^2 du$.

Ito Lemma: 对Ito过程$X_t$, 有
$df(t,X_t) = f_t(t,X_t)dt + f_x(t,X_t)dX_t+\frac{1}{2}f_{xx}(t,X_t)dX_tdX_t$.
特别地, $df(W(t)) = f'(W_t)dW_t + \frac{1}{2}f''(W_t)dt$.

Stock Price $S_t$ following log-normal Brownian motion:
$dS_t = \alpha S_t dt + \sigma S_t dW_t$,
$d \log S_t = (\alpha -\frac{1}{2}\sigma^2)dt+\sigma dW_t$.

Girsanov Thm.
定义$Z(t)=\exp\left(-\int_0^t \Theta(u)dW(u)-\frac12 \int_0^t \Theta(u)^2 d\right)$,
$\widetilde{W}(t)=W(t)+\int_0^t\Theta(u)du$,
并假设$\Theta(u)Z(u)$平方可积.
令$Z=Z(T)$, 则$\mathbb{E}Z=1$,
且$Z$作为Radon-Nikodym导数, 即$\widetilde{\mathbb{P}}(A)=\int_A Z(\omega)d\mathbb{P}(\omega)$定义的
$\widetilde{\mathbb{P}}$下, $\widetilde{W}(t)$是BM.

在$\Theta(t)=\frac{\alpha(t)-R(t)}{\sigma(t)}$定义的$\widetilde{\mathbb{P}},\widetilde{W}$下,
$dS(t)=R(t)S(t)dt+\sigma(t)S(t)d\widetilde{W}(t)$, 贴现后
$dS^*(t)=\sigma(t) S^*(t) d\widetilde{W}(t)$是鞅.

Martingale Representation Thm.
$M(t)$是平方可积鞅, 则存在适应的平方可积过程$\theta(t)$使得$M(t)$ a.s.有积分表示
$M(t)=\mathbb{E}[M_0]+\int_0^t \theta(s) dW(s)$.

贴现后的 deriv value process $V^*(t)$ 是鞅, 由鞅表示定理有$\theta(t)$使
$V^*(T)=\widetilde{\mathbb{E}}V_0 + \int_0^T \theta(t) d\widetilde{W}(t)$
$=\widetilde{\mathbb{E}}V_0 + \int_0^T \frac{\theta(t)}{\sigma(t)S^*(t)} dS^*(t)$,
即为复制策略.

\subsection{Stochastic Differential Equation}

\textbf{线性SDE求解}  考虑$\left\{\begin{array}{l}
		d X(t)=(c(t)+d(t) X(t)) d t+(e(t)+f(t) X(t)) d W(t) \\
		X(0)=X_0\end{array}\right. $
其中 $c,d,e,f$ 非随机.
结论: 解为 $X_1(t) \cdot\left(X_0+\int_0^t(c(s)-f(s) e(s))\left(X_1(s)\right)^{-1} d s
	+\int_0^t e(s) \left(X_1(s)\right)^{-1} d W(s)\right)$,
其中 $X_1(t)=\exp \left(\int_0^t f(s) d W(s)-\int_0^t (d(s)-\frac{1}{2} f(s^2)) d s\right)$.

解法: Duhamel原理.

先得到 $\left\{\begin{array}{l}d X_1(t)=d(t) X_1(t) d t+f(t) X_1(t) d W(t) \\ X_1(0)=1\end{array}\right.$
的解$X_1(t)$ (对$d\left(\ln X_1(t)\right)$用Ito即可).

再分离变量, 设 $X(t)=X_1(t) \cdot X_2(t)$ 其中$X_2$满足 $\left\{\begin{array}{l}d X_2(t)=A(t) d t+B(t) d W(t) \\ X_2(0)=X_0\end{array}\right.$
是原方程的解, $A(t),B(t)$待定.
写开$dX(t)$代回原方程, 有
$\left\{\begin{array}{l}
		A(t)=(c(t)-f(t) e(t))\left(X_1(t)\right)^{-1} \\
		B(t)=e(t)\left(X_1(t)\right)^{-1}
	\end{array}\right.$, 积分即得$X_2$. 从而有原方程解$X_1\cdot X_2$.

\textbf{Feynman Kac Theorem}
考虑方程 $dX(u)=\beta(u,X(u))du+\gamma(u,X(u))dW(u)$,
记$g(t,x)=\mathbb{E}^{t,x}[h(X(T))]$, 其中$X$是初值$X(t)=x$下的解.
则$g_t+\beta g_x+\frac12 \gamma^2 g_{xx}=0$.

\section{OLS}
$y=X\beta+\varepsilon$, G-M假设 $\text{Cov}(\varepsilon)=\sigma ^2 I_n$.

估计$\hat{\beta}=(X^TX)^{-1}X^Ty$.
满足$\mathbb{E}[\hat{\beta}]=\beta, \operatorname{Var}(\hat{\beta})=\sigma^2(X^TX)^{-1}$.
是正态分布.
是BLUE最佳线性无偏估计.

$X$已知非随机, $y$已知随机, $\beta$未知非随机, $\hat{\beta}$随机, $\varepsilon$未知随机($\sigma^2$ 未知).
每个$y_i$都有一个对应的总体$N(\beta^T x_i, \sigma^2)$

记$H=X(X^TX)^{-1}X^T$,有$\hat{y}=Hy$. $H$对称幂等,$HX=X$,迹为$k+1$.

$\mathbb{E}[\hat{y}]=y, \operatorname{Var}(\hat{y})=\sigma^2H$.
残差$e=y-\hat{y}$(注意区别于误差向量$\varepsilon$)有
$\mathbb{E}[e]=0, \operatorname{Var}(e)=\sigma^2 (I-H)$.

$ESS = SSR = \sum (\hat{y_i}-\bar{y})^2, df=p-1$.
$RSS = SSE = \sum (\hat{y_i}-y_i)^2, df=n-p$.
$SST = \sum (y_i-\bar{y})^2$.
有$SST=SSE+SSR$.定义$R^2=1-SSR/SST$.
(ESS: explained, SSE: error, SSR: regression, RSS: residual).

$\operatorname{Var}(\hat{\beta_j}) = \frac{\sigma^2}{SST_j(1-R_j^2)}$,
其中$SST_j=\sum_i (x_{ij}-\bar{x_j})^2$, $R_j^2$为$x_j$关于其他自变量含截距回归的$R^2$.

MLE: $\hat{\sigma^2}=\frac{RSS}{n}$是有偏的,
无偏估计为$\hat{\sigma^2}=\frac{RSS}{n-k-1}$(或写为$se^2$).

$\hat{\beta_j}$标准误(标准化残差)$se(\hat{\beta_j}) = \frac{\hat{\sigma}}{(SST_j(1-R_j^2))^{1/2}}$.
$SE(\hat{\beta_j})$是$Cov(\hat{\beta})$对应对角元的开根.
T化残差为原始残差除以估计的标准差.


t检验统计量$t_{\hat{\beta_j}} = \hat{\beta_j}/se(\hat{\beta_j})$.
整体回归F统计量$F=\frac{R^2/k}{(1-R^2)/(n-k-1)}$.

衡量多重共线性$VIF_j = \frac{1}{1-R_j^2}$, 大于10认为严重.
有$SE(\beta_j)=\frac{\sigma^2}{(n-1)\hat{Var}(X_j)}VIF_j$.

$y$对$x$做一元回归, 回归系数$\hat{\beta}_{yx}=\frac{Cov(x,y)}{Var(x)}$. 回归$R^2$的分布$\sim Beta(\frac{k-1}{2},\frac{n-k}{2})$. $Beta(\alpha,\beta)$的期望为$\frac{1}{1+\beta/\alpha}$.


\section{Finance}

\subsection{Pricing}

Binomial Model.
风险中性概率 $\widetilde{p} = \frac{1+r-d}{u-d}$, $\widetilde{q} = \frac{u-1-r}{u-d}$.
$V_n(\omega)=\frac{1}{1+r}[\widetilde{p}V_{n+1}(\omega H) + \widetilde{q}V_{n+1}(\omega T)]$.
$\Delta_n(\omega) = \frac{V_{n+1}(\omega H)-V_{n+1}(\omega T)}{S_{n+1}(\omega H)-S_{n+1}(\omega T)}$.


Option value wrt strike price is convex (consider the payoff of fly).
Option value wrt vol is usually (but not always) convex (see Volga expression).

当total variance $\sigma^2 T$较小, ATM put 价值有近似 $P_{ATM}\simeq 0.4\sigma S_0\sqrt{T}$.

Put-Call Parity $C(t)-P(t) = S(t) - Ke^{-r(T-t)}$.

BS equation: 假设股价符合几何布朗运动$dS=\mu S d t+\sigma S dW$, 即$S_t = S_0 \exp ((r-\frac{1}{2}\sigma^2)t+\sigma W(t))$.
$\frac{\partial V}{\partial t}+\frac{1}{2} \sigma^{2} S^{2} \frac{\partial^{2} V}{\partial S^{2}}+r S \frac{\partial V}{\partial S}-r V=0$
(推导: 对$e^{-rt}V(t,S(t))$用Ito).

BSM formula:
$c=S_0 N\left(d_1\right)-K \mathrm{e}^{-r T} N\left(d_2\right) $,
$p=K \mathrm{e}^{-r T} N\left(-d_2\right)-S_0 N\left(-d_1\right) $, 其中
$d_1=\frac{\ln \left(S_0 / K\right)+\left(r+\sigma^2 / 2\right) T}{\sigma \sqrt{T}} $,
$d_2=\frac{\ln \left(S_0 / K\right)+\left(r-\sigma^2 / 2\right) T}{\sigma \sqrt{T}}=d_1-\sigma \sqrt{T}$.


\subsection{Greeks}
Delta: 关于Spot一阶导.
Gamma: 关于Spot二阶导.
Vega: 关于Vol一阶导.
Theta: 关于t一阶导.
Rho: 关于利率一阶导.

\noindent \textbf{Delta \& Gamma} \par
\begin{itemize}
	\item Call的Delta是正的, Put的Delta是负的. Call和Put的Gamma都是正的.
	\item 对Call, 随着Vol下降, ITM Delta上升(范围从50到100), OTM Delta下降;
	      随着到期日临近, ITM Delta上升, OTM Delta下降.
	      波动率下降/到期日临近下,Delta-标的价格变化图像为Fig1左.
	\item 对Put, 随着波动率上升, OTM Delta下降(范围从0到-50), ITM Delta上升;
	      随着到期日临近, ITM Delta下降, OTM Delta上升.
	\item Gamma在ATM最大.ATM的Gamma与Strike负相关.
	\item 随波动率下降/到期日临近下, 期权的Gamma值变化图像为Fig1右.
	      \begin{figure}[H]
		      \centering
		      \includegraphics[width=0.8\textwidth]{fig/fig1.png}
		      \caption{Fig1}
	      \end{figure}

\end{itemize}

\noindent \textbf{Theta} \par
\begin{itemize}
	\item Theta通常是负的.
	\item Theta值在平值时绝对值最大, 深度实值或深度虚值时绝对值较小.
	\item 期权价值与|Theta|在临近到期时的变化见Fig2.
	      \begin{figure}[H]
		      \centering
		      \includegraphics[width=0.7\textwidth]{fig/fig2.png}
		      \caption{Fig2}
	      \end{figure}
	\item 平值期权的Theta与波动率成比例变化.
\end{itemize}


\noindent \textbf{Vega} \par
\begin{itemize}
	\item Vega在平值时最大, 深度实值或深度虚值时较小.
	\item 期权价值与Vega在不同波动率下的情况见Fig3.
	      \begin{figure}[H]
		      \centering
		      \includegraphics[width=0.7\textwidth]{fig/fig3.png}
		      \caption{Fig3}
	      \end{figure}
	\item Vega值随到期日临近的变化见Fig4.
	      \begin{figure}[H]
		      \centering
		      \includegraphics[width=0.4\textwidth]{fig/fig4.png}
		      \caption{Fig4}
	      \end{figure}

\end{itemize}


\subsection{Structure}


\noindent \textbf{对称策略} \par

跨式期权 Straddle: Long 100 C + Long 100 P (\textbackslash /).
Long Gamma, Long Vega, Short Theta.

宽跨式期权 Strangle: Long 105 C + Long 95 P (\textbackslash \_ / ).
Long Gamma, Long Vega, Short Theta.

蝶式期权 Fly: Long 95 C + Long 105 C + Short 2 * 100 C (\_ / \textbackslash \_).
Short Gamma, Short Vega, Long Theta.

鹰式期权 Iron Condor: Long 95 C + Long 105 C + Short 97.5 C + Short 102.5 C (\_ / - \textbackslash \_).
Short Gamma, Short Vega, Long Theta.

\noindent \textbf{时间价差} \par

时间价差: 买入长期期权, 卖出短期期权(长期期权通常更贵).
使用平值期权1:1构成Delta中性.

Short Gamma, 因为若标的价格不变, 随时间流逝, 短期期权价值减少是多于长期期权的.

Long Vega, 因为波动率变化(上升)对长期期权有更大影响(增加更多时间价值).

\begin{figure}[H]
	\centering
	\includegraphics[width=0.7\textwidth]{fig/fig5.png}
	\caption{Fig5}
\end{figure}

策略选择: 如果隐含波动率低, 要 Long Vega.
当隐含波动率很低但被认为会升高时, 时间价差多头可能获利.

\noindent \textbf{垂直价差} \par

买入与卖出相同类型、同时到期、行权价格不同的期权.

Call Spread: Buy 105 C + Sell 110 C.
Put Spread: Buy 95 P + Sell 90 P.

买低卖高, 无论是Call还是Put都具有牛市特征, 即具有正的Delta值.
两个行权价之间差距越大, 牛市特征越明显(价差的Delta越大).

Call Spread Collar: CS + underwrite.
Put Spread Collar: PS + overwrite.

Risk reversal: Buy downside put, partially funded by selling an upside call, or vice versa.

\begin{figure}[H]
	\centering
	\includegraphics[width=0.7\textwidth]{fig/fig6.png}
	\caption{Fig5}
\end{figure}


\subsection{其他}

\noindent \textbf{美式期权} \par
看涨期权价值=内在+波动率+利率-股利. 因此美式看涨若提前行权, 只可能在股利支付前一天.
看跌期权价值=内在+波动率-利率+股利. 因此美式看跌若提前行权, 要距离除息日足够远, 且波动率价值较小.

严格说明在 no dividend 情况下 Am Call 不会 early exercise:
考虑欧式期权Put-Call Parity, 有 $C_{eu} = P_{eu} + S - Ke^{-r(T-t)} > P + S - K > S - K$,
从而 $C_{am} \geq C_{eu}>S-K$.

无论看涨看跌, 越是实值, 美式与欧式价格差越大; 波动率越小, 美式与欧式价格差越大.

美式期权没有Put Call Parity, 没有显式定价公式(Perpetual American Option有).


\noindent \textbf{Exotic} \par
Down and out + Down and in = Vanilla Call.

Knock-in: Structure is activated after being knocked in (before that coupon is protected).

Knock-out: Immediate termination of structure.

Autocall Hedging: 
The Vega of autocall peaks somewhere between KI and spot, with vega at KI only about 20\% peak vega.
When peak vega is hit, need to hedge skew exposure (sell vol around peak vega and buy vol around KI).
Spot near KO, hedge with buying tight call spreads (digital payoff).
Spot near KI, buy back all hedges.


\noindent \textbf{Bond and Rates} \par

$n$年零息利率 spot rate / zero rate: 此时投入本金, 所有利息与本金都在$n$年后支付的收益率.

平价收益率(par yield): 债券每期支付券息$c$, 所有现金流(每期的$c$与到期的本金加$c$)用零息利率贴现等于债券面值(100).

若$d$为到期时\$1贴现值, $A$为年金(annuity, 每个券息日付\$1)现金流贴现值, $m$为每年付息次数, 则$100=A\frac{c}{m}+100d$.

债券收益率$y$: 所有现金流用$y$进行贴现, 现值等于债券市场价格.

久期$D=-\frac{1}{B}\frac{\partial B}{\partial y}$, 其中$B$为债券价格, $y$为收益率. 刻画了债券价格关于收益率的敏感度: $\Delta B\simeq -DB\Delta y$.

Zero Coupon Bond的久期等于它的maturity.


\noindent \textbf{VaR} \par

$N$ day $C\%$ VaR has $VaR(N,C)\simeq \sigma_v z_C \sqrt{\frac{N}{252}}V(0)$,
where $C$ is the confidence level, $\sigma_v$ is the annualized std of return, $z_C$ is the z-score i,e, $\mathbb{P}(Z\leq z_C)=C$ for std norm $Z$.
$V(0)$ is the current value of portfolio.

\section{Linear Algebra}

\subsection{矩阵性质}

\begin{itemize}

	\item $|\boldsymbol{A B}|=|\boldsymbol{A}||\boldsymbol{B}|$, $|k\boldsymbol{A}|=k^n|\boldsymbol{A}|$

	\item $(k \boldsymbol{A})^{-1}=k^{-1} \boldsymbol{A}^{-1}$, $|\boldsymbol{A}^{-1}|=|\boldsymbol{A}|^{-1}$

	\item $\operatorname{tr}(\boldsymbol{A} \boldsymbol{B})=
		      \operatorname{tr}(\boldsymbol{B} \boldsymbol{A})$,
	      $\operatorname{tr}\left(\boldsymbol{A} \boldsymbol{A}^{\prime}\right)
		      \geq \operatorname{tr}\left(\boldsymbol{A}^2 \right)$.

	\item 若 $\boldsymbol{A}$ 是 $n$ 阶实矩阵,
	      则 $\operatorname{tr}\left(\boldsymbol{A} \boldsymbol{A}^{\prime}\right) \geq 0$,
	      等号成立$\Leftrightarrow\boldsymbol{A}=\boldsymbol{O}$\par

	\item $r(\boldsymbol{A} \boldsymbol{B}) \leq \min \{r(\boldsymbol{A}), r(\boldsymbol{B})\}$;\par

	\item $r(\boldsymbol{A}+\boldsymbol{B}) \leq r(\boldsymbol{A})+r(\boldsymbol{B}),
		      r(\boldsymbol{A}-\boldsymbol{B}) \leq r(\boldsymbol{A})+r(\boldsymbol{B})$;\par

	\item Sylvester不等式.
	      设$\boldsymbol{A}$是$m\times n$矩阵,$\boldsymbol{B}$是$n\times t$矩阵,则
	      $r(\boldsymbol{A} \boldsymbol{B}) \geq r(\boldsymbol{A})+r(\boldsymbol{B})-n$.

	\item 设$\boldsymbol{A}$是$m\times n$阶实矩阵,则
	      $r\left(\boldsymbol{A}^{\prime} \boldsymbol{A}\right)=
		      r\left(\boldsymbol{A} \boldsymbol{A}^{\prime}\right)=r(\boldsymbol{A})$

	\item 设$\boldsymbol{A}$是$m\times n$矩阵,则:\par
	      $(1)$若$r(\boldsymbol{A})=n$,即$\boldsymbol{A}$列满秩,则必存在秩等于$n$的
	      $n\times m$矩阵$\boldsymbol{B}$使$\boldsymbol{B}\boldsymbol{A}=\boldsymbol{I}_{n}$;\par
	      $(2)$若$r(\boldsymbol{A})=m$,即$\boldsymbol{A}$行满秩,则必存在秩等于$m$的
	      $n\times m$矩阵$\boldsymbol{C}$使$\boldsymbol{AC}=\boldsymbol{I}_{m}$.

	\item 满秩分解.
	      设$\boldsymbol{A}$是秩为$r$的$m\times n$矩阵,则有满秩分解$\boldsymbol{A}=\boldsymbol{B}\boldsymbol{C}$,
	      其中$\boldsymbol{B}$是秩为$r$的$m\times r$矩阵,$\boldsymbol{C}$是秩为$r$的$r\times n$矩阵.

	\item 设$A$是$n$阶幂等矩阵,则存在n阶非异阵$\boldsymbol{P}$,使得$\boldsymbol{P}^{-1} \boldsymbol{A P}=\left(\begin{array}{ll}\boldsymbol{I}_{r} & \boldsymbol{O} \\ \boldsymbol{O} & \boldsymbol{O}\end{array}\right)$,
	      其中$r=\mathrm{r}(\boldsymbol{A})$

\end{itemize}


\subsection{特征值}

\begin{itemize}

	\item 任一复方阵必相似于一上三角阵,且主对角线上元素为特征值.

	\item Cayley-Hamilton Thm. $f(x)$为矩阵$A$的特征多项式, 则$f(A)=O$.

	\item 若$A$特征值为$\lambda_1,\cdots,\lambda_n$, 则$f(A)$的特征值为$f(\lambda_1),\cdots,f(\lambda_n)$.
	      若$A$适合多项式$g(x)$,则$A$的任意特征值适合$g(x)$.

	\item 特征值降阶公式.
	      $A, B$是$m \times n$与$n \times m$矩阵, $m \geq n$.则
	      $\left|\lambda I_{m}-A B\right|=\lambda^{m-n}\left|\lambda I_{n}-B A\right|$.

	\item 可对角化判定方法\par
	      充分条件: 有 $n$ 个不同的特征值; 相似于实对称阵.\par
	      充要条件: 有 $n$ 个线性无关的特征向量; 极小多项式无重根;
	      初等因子都是一次多项式; Jordan块都是一阶矩阵;
	      有完全的特征向量系 (即$m$重特征值有$m$个线性无关特征向量).

	\item Jordan标准型求法\par
	      $\lambda I-A$相抵于$\operatorname{diag}\{1,\dots,1,d_1(\lambda),\dots,d_k(\lambda)\}$, 对角元$d_i | d_{i+1}$.
	      $d_1,\dots,d_k$的准素因子全体(初等因子)对应Jordan块拼起来即为Jordan标准型.

	\item $J=J_r(\lambda )$为Jordan块.则$f(J)$主对角线上均为$f(\lambda )$, 上次对角线上均为$f'(\lambda)$,
	      距离主对角线$j$位的对角线上均为$\frac{1}{j!}f^{(j)}(\lambda)$.
	      对$\lambda\neq 0$, $J_r(\lambda)^m$的Jordan标准型为$J_r(\lambda^m)$.

	\item $A^k$求法: 先求出Jordan标准型$J$, 根据$AP=PJ$解方程得$P$, 求$J^k$再乘过渡矩阵即可.

\end{itemize}


\subsection{二次型}

\begin{itemize}
	\item 设$A$为$n$阶实对称阵,则以下结论等价:\par
	      $(1)$$A$正定\quad $(2)$$A$合同于$I_n$\quad $(3)$存在非异实矩阵$C,A=C'C$\quad $(4)$$A$的主子式都大于$0$\par
		      $(5)$$A$的顺序主子式都大于$0$\quad $(6)$$A$的特征值都大于$0$.

	\item $\boldsymbol{A}=\left(a_{i j}\right)$是$n$阶半正定实对称矩阵,则$|\boldsymbol{A}| \leq a_{11} a_{22} \cdots a_{n n}$,
	      等号成立当且仅当 $\boldsymbol{A}$ 是对角矩阵或某个$a_{ii}=0$.

	\item 设$\boldsymbol{A}$为$n$阶实对称矩阵, 则$\boldsymbol{A}$半正定或半负定$\Leftrightarrow$
	      对任意$\boldsymbol{\alpha}$满足$\boldsymbol{\alpha}^{\prime} \boldsymbol{A} \boldsymbol{\alpha}=0$,
	      均有$\boldsymbol{A} \boldsymbol{\alpha}=\mathbf{0}$.

	\item 若实矩阵$A$满足$A^{\prime}=A^{-1}$, 则称$A$为正交阵.
	      A为正交阵/酉阵的充要条件为$A$的$n$个行向量构成一组标准正交基.

	\item 实对称阵与Hermite阵特征值全为实数,反对称阵特征值全为0或纯虚数.

	\item 实对称阵与Hermite阵合同于$\operatorname{diag}\{I_p,-I_q,0\}$.
	      实对称阵(Hermite阵)正交相似(酉相似)于实对角阵.

	\item 复正规阵酉相似于复对角阵,对角元模长都为1.实反对称阵酉相似于复对角阵.

	\item Gram-Schmidt正交化方法.
	      $\{u_1,\dots,u_m\}$为一组线性无关的向量,
	      令$$v_{k+1}=u_{k+1}-\sum _{i=1}^k\frac{(u_{k+1},v_i)}{\Vert v_i\Vert^2}v_i$$
	      则$\{v_1,\dots,v_m\}$为一组正交向量.

\end{itemize}

\subsection{一些分解}

\begin{itemize}

	\item QR分解\par
	      设$A$为$n$阶实/复矩阵, 则$A=QR$, 其中$Q$为正交/酉阵, $R$为主对角元均$\geq 0$的上三角阵. 若$A$非异, 则分解唯一.

	\item Jordan-Chevalley分解\par
	      设$A$为$n$阶复方阵, 则有分解$A=B+C$,满足:\par
	      $(1)$$B$可对角化\qquad $(2)$$C$为幂零阵\qquad $(3)$$BC=CB$\qquad $(4)$$B,C$可表示为$A$的多项式\par
	      并且满足前三条的分解唯一.

	\item Cholesky分解\par
	      设$A$为$n$阶实对称阵/Hermite阵, 若$A$正定, 则存在对角元均为正实数的下三角(或上三角)实矩阵$L$
	      使得$A=LL^{\prime}$/$A=L\overline{L^{\prime}}$, 并且分解唯一.
	      若$A$为半正定阵, 则对角元均为非负数, 且分解不一定唯一.

	\item 极分解\par
	      $\varphi \in \mathcal{L}(V)$, 则$\varphi = \omega\psi$. 其中$\omega$保积,$\psi$半正定自伴随.若$\varphi $可逆,则分解唯一. \par
	      $A\in M_n(R)$, 则$A=OS$, 其中$O$为正交阵, $S$为半正定实对称阵.\par
	      $A\in M_n(C)$, 则$A=UH$, 其中$U$为酉阵, $H$为半正定Hermite阵.

	\item 奇异值分解\par
	      $A\in M_{m\times n}(R)$,则$A=P \operatorname{diag}\{S,0\}Q^{\prime}$.
	      其中$P,Q$为$m,n$阶正交阵,$S=\operatorname{diag}\{\sigma_1,\cdots ,\sigma_r\}$,
	      $\sigma_1\geq \cdots \geq \sigma_r>0$为$A$的所有奇异值,即$\sigma_1^2,\cdots,\sigma_r^2$为$A^{\prime}A$的非零特征值全体.

\end{itemize}


\subsection{计算}

初等变换法求逆阵: $A$是非异阵, 作$n\times 2n$矩阵$(A, I_n)$, 实施初等行变换,
将左侧化为$I_n$, 此时右侧矩阵即为$A^{-1}$.

$A$是实对称阵, 求非异矩阵$C$使得$C'AC$是对角阵:
作$n\times 2n$矩阵$(A, I_n)$, 实施初等行变换, 并对左侧矩阵实施对应初等列变换,
将左侧化为对角阵. 此时右侧矩阵的即为$C'$.

$A$是实对称阵, 求正交矩阵$P$使得$P'AP$是对角阵:
求特征值, 对每个特征值$\lambda$, 求解$(\lambda I-A)x=0$得到基础解系.
若基础解系有多个向量, 进行正交化.
所有列向量单位化后拼接得到$P$.

\section{Calculus}

\subsection{一元}

\noindent \textbf{等价量代换}  \quad 在$x\to0$时, \\
$(1+ax)^b\sim 1+abx,$
$\cos x\sim 1-\frac{1}{2}x^2$, 
$x-\sin x\sim \frac{1}{6}x^3,x-\tan x\sim -\frac{1}{3}x^3,
\tan x-\sin x\sim \frac{1}{2}x^3$, \\
$\log_a(1+x)\sim \frac{x}{\ln a},a^x-1\sim x\ln a$.


\noindent \textbf{常用导数}\\
$(\tan x)^{\prime}=\sec ^{2} x $\qquad $ (\cot x)^{\prime}=-\csc ^{2} x $\qquad
$(\sec x)^{\prime}=\tan x \sec x $\qquad $ (\csc x)^{\prime}=-\cot x \csc x $\\
$(\arcsin x)^{\prime}=\frac{1}{\sqrt{1-x^{2}}}$
\qquad $(\arccos x)^{\prime}=-\frac{1}{\sqrt{1-x^{2}}}, x \in(-1,1) $\\
$(\arctan x)^{\prime}=\frac{1}{1+x^{2}} $
\qquad $(\operatorname{arccot} x)^{\prime}=-\frac{1}{1+x^{2}} $\\ 
$\left(\log _{a} x\right)^{\prime}=\frac{1}{x \ln a}$\qquad $\left(a^{x}\right)^{\prime}=a^{x} \ln a$


\noindent \textbf{常用高阶导数}\\
$\left(x^{k}\right)^{(n)}=k(k-1) \cdots(k-n+1) x^{k-n} \quad(n \leq k)$,\\
$\left(a^{x}\right)^{(n)}=a^{x}(\ln a)^{n}$,
$(\ln x)^{(n)}=(-1)^{n-1}(n-1) ! x^{-n}$, \\
$(\sin x)^{(n)}=\sin \left(x+\frac{n \pi}{2}\right),
(\cos x)^{(n)}=\cos \left(x+\frac{n \pi}{2}\right)$


\noindent \textbf{分部积分法}
$\int u(x) v^{\prime}(x) \mathrm{d} x= u(x) v(x)-\int v(x) \mathrm{d} u(x) =u v-\int v(x) u^{\prime}(x) \mathrm{d} x$
\par 按照$u$在左,$v$在右的次序:反对幂指三


\noindent \textbf{常用变量代换}\par 
含有$\sqrt[n]{\frac{ax+b}{cx+d}}$的积分,令$t=\sqrt[n]{\frac{a x+b}{c x+d}}$;\par 
含有$\sqrt{x^{2}+a^{2}}$的积分,令$x=a \tan t$或$x=a \cot t$;\par 
含有$\sqrt{x^{2}-a^{2}}$的积分,令$x=a \sec t$或$x=a \csc t$;\par 
含有$\sqrt{a^{2}-x^{2}}$的积分,令$x=a \sin t$或$x=a \cos t$;\par
三角有理函数积分 $\int R(\sin x, \cos x) dx$\par 
$R(-\sin x, \cos x)=-R(\sin x, \cos x)$,令 $t=\cos x $\par 
$R(\sin x,-\cos x)=-R(\sin x, \cos x)$,令 $t=\sin x $\par 
$R(-\sin x,-\cos x)=R(\sin x, \cos x)$,令$t=\tan x$


\noindent \textbf{常用积分}\\
$\int\tan x \mathrm{~d} x=-\ln |\cos x|+C,\int\cot x \mathrm{~d} x=\ln |\sin x|+C $\\
$\int\csc x\mathrm{~d}x=\ln|\csc x-\cot x|+C,\int\sec x \mathrm{~d} x=\ln |\sec x+\tan x|+C $\\
$\quad \int a^{x} \mathrm{~d} x=\frac{1}{\ln a} a^{x}+C $, $\int \ln x \mathrm{~d} x= x \ln x -x +C$\\
$\int \frac{1}{\sqrt{a^2-x^{2}}} \mathrm{~d} x=\arcsin\frac{x}{a}+C $,
$\int \frac{1}{\sqrt{x^{2}\pm a^{2}}} \mathrm{~d} x=\ln \left|x+\sqrt{x^{2}\pm a^{2}}\right|+C$ \\
$\int \frac{1}{a^2+x^{2}} \mathrm{~d} x=\frac{1}{a}\arctan\frac{x}{a}+C $,
$\int \frac{1}{a^2-x^2}\mathrm{~d}x=\frac{1}{2a}\ln|\frac{a+x}{a-x}|+C $\\
$\int \sin ^{2} x \mathrm{~d} x=\frac{1}{2}(x-\sin x \cos x)+C ,\int \cos ^{2} x \mathrm{~d} x=\frac{1}{2}(x+\sin x \cos x)+C$\\
$\int \tan^2(x)\mathrm{~d} x=\tan x-x+C,\int\cot^2(x)\mathrm{~d} x=-\cot x-x+C$ \\
有理函数不定积分: 化为简单分式$\frac{A}{(x-a)^{k}}, \frac{A x+B}{\left((x-a)^{2}+b^{2}\right)^{k}}$的和, 然后
$\int \frac{A x+B}{\left((x-a)^{2}+b^{2}\right)^{k}} \mathrm{d} x 
= \int \frac{A(x-a)}{\left((x-a)^{2}+b^{2}\right)^{k}} \mathrm{d} x
+\int \frac{B+a A}{\left((x-a)^{2}+b^{2}\right)^{k}} \mathrm{d} x $, 
$\int \frac{1}{\left((x-a)^{2}+b^{2}\right)^{k}} \mathrm{d} x \stackrel{t=\frac{x-a}{b}}{=}
\frac{1}{b^{2 k-1}} \int \frac{1}{\left(t^{2}+1\right)^{k}} \mathrm{d} t$, \\
$\int \frac{1}{\left(x^{2}+1\right)^{2}} \mathrm{~d} x=\frac{\arctan(x)}{2}+\frac{x}{2+2x^2}+C$\\

\noindent Stirling公式. $n!\sim \sqrt[]{2\pi n}\left(\frac{n}{e}\right)^n (n\to\infty )$\\
Viete公式. $\prod_{n=1}^{\infty} \cos (\frac{x}{2^n})=\frac{\sin x}{x} $\\
Wallis公式. $\prod_{n=1}^{\infty} (1-\frac{1}{(2n)^2})=\frac{2}{\pi } $\\


\subsection{多元}

\noindent \textbf{多元微分}

Hessian matrix: $f$为$\mathbb{R}^n\to\mathbb{R}$函数,
$H(f)=\left(\frac{\partial^2 f}{\partial x_i \partial x_j}\right)_{n\times n}$

Jacobi matrix: $f$为$\mathbb{R}^n\to \mathbb{R}^m$向量值函数,
$J(f)=\left(\frac{\partial f_i}{\partial x_j}\right)_{m\times n}$

Prop. $H(f(x))=J(\nabla f(x))^T$.

Hessian矩阵正定的极值点是极小值点, 负定是极大值点.

隐函数存在定理:
对多元函数$F(x_1,\dots,x_n,y)=0$, 在周围闭矩形连续且有连续偏导, $F_y\neq 0$, 有$\frac{\partial y}{\partial x_i}=-\frac{F_{x_i}}{F_y}$.
多元向量值函数$F(x_0,y_0,u_0,v_0)=0, G(x_0,y_0,u_0,v_0)=0$, Jacobi行列式$\frac{\partial(F,G)}{\partial(u,v)}\neq 0$,
则有$\begin{pmatrix} u_x & u_y \\ v_x  & v_y \end{pmatrix} = -\frac{\partial(F,G)}{\partial(u,v)} ^{-1}\frac{\partial(F,G)}{\partial(x,y)}$.


\noindent \textbf{重积分}

重积分变量代换. 映射$T:
	\left\{\begin{matrix}
		x=x(u,v) \\y=y(u,v)
	\end{matrix}\right.,D\to T(D)$
有连续偏导且$\frac{\partial(x,y)}{\partial(u,v)}\neq 0$.
$f(x,y)$在$T(D)$连续, 则
$\iint_{T(D)}f(x,y)\mathrm{d}x\mathrm{d}y=\iint_{D} f(x(u,v), y(u,v))
	\left|\frac{\partial(x,y)}{\partial(u,v)}\right| \mathrm{d}u\mathrm{d}v$


\noindent 常用变量代换:\par
极坐标变换$x=r\cos\theta,y=r\sin\theta$下$\mathrm{d}x\mathrm{d}y=r\mathrm{d}r\mathrm{d}\theta$.\par
柱坐标变换$x=r\cos\theta,y=r\sin\theta,z=z$下$\mathrm{d}x\mathrm{d}y\mathrm{d}z=r\mathrm{d}r\mathrm{d}\theta\mathrm{d}z$.\par
球坐标变换$x=r\sin\varphi\cos\theta,y=r\sin\varphi\sin\theta,z=r\cos\varphi,
	r\in[0,+\infty),\varphi\in[0,\pi],\theta\in[0,2\pi]$下$\mathrm{d}x\mathrm{d}y\mathrm{d}z
	=r^2\sin\varphi \mathrm{d}r\mathrm{d}\varphi \mathrm{d}\theta$.\par
广义球坐标变换$x=ar\sin\varphi\cos\theta,y=br\sin\varphi\sin\theta,z=cr\cos\varphi$下$\mathrm{d}x\mathrm{d}y\mathrm{d}z
	=abcr^2\sin\varphi \mathrm{d}r\mathrm{d}\varphi \mathrm{d}\theta$.\par
n重球坐标变换$x_{1}=r \cos \varphi_{1}, x_{2}=r \sin \varphi_{1} \cos \varphi_{2}, \cdots,
	x_{n-1}=r \sin \varphi_{1}\cdots \sin \varphi_{n-2} \cos \varphi_{n-1},
	x_{n}=r \sin \varphi_{1}\cdots\sin \varphi_{n-1},
	0\leq \varphi_1,\cdots,\varphi_{n-2}\leq \pi,0\leq \varphi_{n-1}\leq 2\pi$下$
	\frac{\partial\left(x_{1}, x_{2}, \cdots, x_{n}\right)}{\partial\left(r, \varphi_{1}, \cdots, \varphi_{n-1}\right)}=
	r^{n-1} \sin ^{n-2} \varphi_{1}$
$\sin ^{n-3} \varphi_{2} \cdots \sin \varphi_{n-2} $. \\


\noindent \textbf{曲线与曲面积分} \par 
\noindent Green公式. $P,Q$在$D$上有连续偏导,则$\int _{\partial D}P\mathrm{d}x+Q\mathrm{d}y=\iint_D (Q_x-P_y)\mathrm{d}x\mathrm{d}y$.
其中沿着边界方向时区域始终在边界左侧.\par 
\noindent Gauss公式. $P,Q,R$在$\Omega $上有连续偏导,
则$\iint _{\partial \Omega}P\mathrm{d}y\mathrm{d}z+Q\mathrm{d}z\mathrm{d}x+R\mathrm{d}x\mathrm{d}y=
\iiint_{\Omega} (P_x+Q_y+R_z)\mathrm{d}x\mathrm{d}y\mathrm{d}z$.其中$\partial \Omega$定向外侧. \par 
\noindent Stokes公式.  $P,Q,R$在$\Sigma$与$\partial \Sigma$上有连续偏导,则
$\int _{\partial \Sigma}P\mathrm{d}x+Q\mathrm{d}y+R\mathrm{d}z=
\iint_{\Sigma} (R_y-Q_z)\mathrm{d}y\mathrm{d}z+(P_z-R_x)\mathrm{d}z\mathrm{d}x+(Q_x-P_y)\mathrm{d}x\mathrm{d}y$.
其中$\partial \Sigma$方向与曲面选定一侧的法向量呈右手关系. \\
\noindent 注.\quad (1)需要连续偏导,不可导点要单独处理.需要闭合,不闭合可以补成闭合.需要连通,环状区域可写成两个连通区域的差.\quad 
(2)由Green公式,若$Q_x=P_y$,则$\int _L P \mathrm{d}x+Q \mathrm{d}y$与路径无关.\\


\noindent \textbf{Euler积分}\par 
\noindent Beta 函数 $B(p,q)=\int _0^1 x^{p-1}(1-x)^{q-1}\mathrm{d}x$\par 
在定义域$(0,+\infty)\times(0,+\infty)$连续.$B(p,q)=B(q,p)$.递推公式$B(p,q)=\frac{q-1}{p+q-1}B(p,q-1)$.\par 
其他表示:$B(p, q)=2 \int_{0}^{\pi / 2} \cos ^{2p-1} \varphi \sin ^{2q-1}\varphi \mathrm{d} \varphi
=\int_0^1 \frac{t^{p-1}+t^{q-1}}{(1+t)^{p+q}} dt$.\par 

\noindent Gamma函数$\Gamma(s)=\int _0^{+\infty}x^{s-1}e^{-x}\mathrm{d}x$ \par 
在定义域$s>0$连续且$n$阶可导,$\Gamma^{(n)}(s)=\int _0^{+\infty}x^{s-1}e^{-x}(\ln x)^n\mathrm{d}x$.
递推公式$\Gamma (s+1)=s\Gamma(s)$.\par 
其他表示:令$x=t^2$,则$\Gamma(s)=2\int _0^{+\infty}t^{2s-1}e^{-t^2}\mathrm{d}t$.

\noindent 相关公式\par 
(1)$B(p,q)=\frac{\Gamma(p)\Gamma(q)}{\Gamma(p+q)}$ \qquad 
(2)Legendre公式\quad $\Gamma(s)\Gamma(s+\frac{1}{2})=\frac{\sqrt{\pi}}{2^{2s-1}}\Gamma(2s)$\par 
(3)余元公式  $\Gamma(s)\Gamma(1-s)=\frac{\pi}{\sin (\pi s)}$ \qquad 
(4)一些取值 $B(\frac{1}{2},\frac{1}{2})=\pi$, $\Gamma(n+1)=n!$, $\Gamma(\frac{1}{2})=\sqrt{\pi }$.


\section{常微}

\noindent 1. 分离变量 \par
(1)齐次方程$\frac{dx}{dt}=f(\frac{x}{t})$. 令$\frac{x}{t}=u$,则$\frac{du}{dt}t=f(u)-u$.\par
(2)$\frac{dx}{dt}=f(at+bx+c)$. 令$y=at+bx+c$,则$\frac{dy}{dt}=a+bf(y)$.\par
~\\

\noindent 2. 非齐次线性方程 $\frac{dx}{dt}=P(t)x+Q(t)$\par
常数变易,两边乘$e^{-\int P(t)dt}$. 常数变易公式$x=e^{\int P(t)dt}(\int e^{-\int P(t)dt}Q(t)dt+C)$.
~\\


\noindent 3. 全微分方程 $M(t,x)dt+N(t,x)dx=0$ \par
(1) $\frac{\partial M(t,x)}{\partial x}=\frac{\partial N(t,x)}{\partial t}$, 则存在$u(t,x)$使得$du(t,x)=M(t,x)dt+N(t,x)dx$\par
设$u(t,x)=\int M(t,x)dt+\varphi(x)$,由$N(t,x)=\frac{\partial }{\partial x}\int M(t,x)dt+\varphi'(x)$可求出$\varphi(x)$. \par
(2) $\frac{\partial M(t,x)}{\partial x} \neq \frac{\partial N(t,x)}{\partial t}$,
要找积分因子$\mu$使得$\frac{\partial \mu M}{\partial x}=\frac{\partial \mu N}{\partial t}$ \par
可寻找特殊形式的$\mu(t,x) $,如取$\mu (x),\mu (t),\mu (xt),\dots $,代入$\frac{\partial \mu M}{\partial x}=\frac{\partial \mu N}{\partial t}$ 求解.
~\\


\noindent 4. 导数未解出的一阶方程 \par
$x=f(t,\frac{dx}{dt})$或$t=f(x,\frac{dx}{dt})$,
引入$p=\frac{dx}{dt}$求解.
~\\


\noindent 5. 高阶方程的降阶 \par
(1)不含$x$,$F\left(t, x^{(k)}, \cdots, x^{(n)}\right)=0$.
令$x^{(k)}=y$,则化为$F\left(t, y, \cdots, y^{(n-k)}\right)=0 $.\par
(2)不含$t$,$F\left(x, \frac{d x}{d t}, \cdots, \frac{d^{n} x}{d t^{n}}\right)=0$.
引入$y=\frac{dx}{dt}$,以$x$为新自变量.则$x^{(k)}$可用$y,\frac{dy}{dx},\cdots,\frac{d^{k-1}y}{dx^{k-1}}$表示.\par
~\\

\noindent 6. 齐次常系数线性方程
$x^{(n)}+a_1x^{(n-1)}+\cdots +a_{n-1}x^{(1)}+a_nx=0$ \par
特征方程$\lambda ^{n}+a_1\lambda^{n-1}+\cdots +a_{n-1}\lambda+a_n=0$.
$\lambda_1,\dots ,\lambda_s$为不同实根,重数$n_1,\dots ,n_s$,
$\alpha_1 \pm i\beta_1,\dots ,\alpha_p \pm i\beta_p$为共轭虚根,重数为$m_1,\dots ,m_p$.
则实值通解$$x(t)=\sum_{i=1}^s P_i(t)e^{\lambda_i t} + \sum_{i=1}^p e^{\alpha_i t}(W_i(t)\cos \beta_i t +V_i(t) \sin\beta_i t)$$ \par
其中$P_i(t)$为$n_i-1$次实系数多项式,$W_i(t),V_i(t)$为$m_i-1$次实系数多项式.
~\\

\noindent 7. 非齐次常系数线性方程
$x^{(n)}+a_1x^{(n-1)}+\cdots +a_{n-1}x^{(1)}+a_nx=f(t)$ \par
解为$x(t)=x_1(t)+x^*(t)$,其中$x_1(t)$为$x^{(n)}+a_1x^{(n-1)}+\cdots +a_{n-1}x^{(1)}+a_nx=0$的通解,
$x^*(t)$为$x^{(n)}+a_1x^{(n-1)}+\cdots +a_{n-1}x^{(1)}+a_nx=f(t)$的特解. \par
$x^*(t)=\int _0^t K(t-s)f(s)ds$,其中$K(t)$为$x^{(n)}+a_1x^{(n-1)}+\cdots +a_{n-1}x^{(1)}+a_nx=0$
在初值条件$K(0)=K'(0)=\cdots =K^{(n-1)}(0)=0, K^{(n)}(0)=1$下的解.
~\\

\noindent 8. Euler方程
$t^nx^{(n)}+a_1t^{n-1}x^{(n-1)}+\cdots +a_{n-1}tx^{(1)}+a_nx=0$ \par
引入$s=\ln |t|$,记$D_s=\frac{d}{ds}$,则$t^n x^{(n)}=D_s(D_s-1)\cdots (D_s-n+1)x$.
~\\

\noindent 9. 齐次常系数线性微分方程组$\frac{d\vec{x}}{dt}=A\vec{x} $ \par

若$A$可对角化,则$\vec{x}(t)=\sum _{i=1}^n c_i e^{\lambda_i t}\vec{p_i}$,$\vec{p_i}$为$\lambda_i$的特征向量. \par
若A有复数特征值$\lambda=\alpha+i\beta$和$\bar{\lambda}=\alpha-i\beta$,
则方程组实值解形如$\vec{x}=e^{\alpha t}\{\vec{p}(t) \cos \beta t+\vec{q}(t) \sin \beta t\}$,
其中$\vec{p}(t)$和$\vec{q}(t)$是$t$的次数小于$\lambda$重数的实向量多项式,其向量系数由微分方程组确定. \par

对一般矩阵$A$,Jordan标准型为$J$,$P^{-1}AP=J$,则$\vec{x}(t)=Pe^{Jt}\vec{c}$,
其中对$\lambda_i$的Jordan块$J_i$,$e^{J_it}$第一行为$e^{\lambda_it},te^{\lambda_it},\dots ,\frac{t^{n_i-1}}{(n-1)!}e^{\lambda_it}$.
记$\vec{p}_i$为$P$的第$i$个列向量,则有 \par
$\vec{x}(t)= \sum_{i=1}^{s}\left\{c_{i_1}e^{\lambda_it} \vec{p}_{i_1}+c_{i_2}\left[te^{\lambda_it} \vec{p}_{i_1}+e^{\lambda_it}\vec{p_{i_2}}\right]
	+\cdots+c_{i_{n_{i}}}\left[\frac{t^{n_{i}-1}}{\left(n_{i}-1\right)!} e^{\lambda i t} \vec{p_{i_1}}+\cdots+e^{\lambda_it} \vec{p}_{i_{n_{i}}}\right]\right\}$.\par
$\forall \operatorname{Re}\{\lambda_i\}<0 \Leftrightarrow \|\vec{x}(t)\| \leq M$.
~\\

\noindent 10. 非齐次常系数线性微分方程组$\frac{d\vec{x}}{dt}=A\vec{x} +\vec{f}(t)$ \par
方程的解$\vec{x}(t)=Pe^{Jt}\vec{c(t)}=Pe^{Jt}\vec{c}+\int_{0}^te^{A(t-s)}\vec{f}(s)ds$ \par
对初值问题$\vec{x}(t_0)=\vec{x_0}$,有$\vec{x}(t)=e^{A(t-t_0)}\vec{x_0}+\int_{t_0}^te^{A(t-s)}\vec{f}(s)ds$.
~\\


\section{Others}

\subsection{矩阵范数}

Frobenius范数$\| A\|=\sqrt{\sum_{i,j}a_{ij}^2}$

由向量范数$\| \vec{x}\| _p=(\sum _i |x_i|^p)^{1/p}$诱导的矩阵范数$\| A\|_p=\operatorname{sup}
	_{\vec{x}\neq \vec{0}} \frac{\| A\vec{x} \|_p}{\| \vec{x} \|_p}$.


$\|A\|_{1}=\max _{1 \leqslant j \leqslant n}\left\{\sum_{i=1}^{n}\left|a_{i j}\right|\right\},
	\|A\|_{2}=\sqrt{\lambda_{\max} \left(A^T A\right)},
	\|A\|_{\infty}=\max _{1 \leqslant i \leqslant n}\left\{\sum_{j=1}^{n}\left|a_{i j}\right|\right\}$. \par

由向量范数诱导的矩阵范数有性质
(1)$\|A \vec{x}\| \leqslant\|A\| \cdot\|\vec{x}\|$ \quad
(2)$\|A\|=\sup _{\|\vec{x}\|=1}\|A \vec{x}\|$ \quad
(3)$\|A B\| \leqslant\|A\| \cdot\|B\|$ \quad
(4)$\left\|\int_{\alpha}^{\beta} A(s) d s\right\| \leqslant\left|\int_{\alpha}^{\beta}\|A(s)\| d s\right|$
~\\

矩阵指数函数: $(e^{At})^{-1}=e^{-At}$, $e^{A(t+s)}=e^{At}\cdot e^{As}$.


\subsection{矩阵求导}
关于$t$求导与积分: 对矩阵每个元素求导与积分.

$\frac{d}{dt}[A(t)B(t)]=A'(t)B(t)+A(t)B'(t)$.

将矩阵$X_{m\times n}$按列堆栈向量化, $vec(X)=[x_{11},\dots,x_{m1},\dots,\dots,x_{mn}]^T$.
将矩阵函数$F:m\times n\to p\times q$向量化, $vec(F(X))=[f_{11}(X),\dots,f_{p1}(X),\dots,\dots,f_{pq}(X)]^T$.

$D_X F(X)=\frac{\partial vec_{pq\times 1}(F(X))}{\partial vec_{mn\times 1}^T X} \in \mathbb{R}^{pq\times mn}$.


向量变元标量函数:
$\frac{\partial a^Tx}{\partial x}=a$,
$\frac{\partial x^TAx}{\partial x}=Ax+A^Tx$,
$\frac{\partial a^T xx^T b}{\partial x}=ab^Tx+ba^Tx$.

矩阵变元标量函数
$\frac{\partial a^T X b}{\partial X}=ab^T$,
$\frac{\partial a^T XX^T b}{\partial X}=ab^TX+ba^TX$.

\subsection{采样}

均匀采正态: Box-Muller变换. $U_1, U_2$是独立的$[0, 1]$均匀分布,
则$Z_0=\sqrt{-2\ln U_1}\cos(2\pi U_2)$, $Z_1=\sqrt{-2\ln U_1}\sin(2\pi U_2)$
是标准正态分布随机变量.

Acceptance-Rejection Sampling.
由给定pdf $f(x)$, 产生服从该分部的样本.
需要建议分布$G$ (pdf $g(y)$ 已知, 通常用均匀/正态)产生候选样本.
计算$c$为满足$cg(x)\geq f(x)$的最小值.
从$G$抽样出样本$Y$, 从均匀分布$U(0,1)$抽样出样本$U$.
如果$U\leq\frac{f(Y)}{cg(Y)}$则接受$Y$, 否则拒绝.


\end{document}
